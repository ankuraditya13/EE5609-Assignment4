
\documentclass[journal,12pt,twocolumn]{IEEEtran}

\usepackage{setspace}
\usepackage{gensymb}

\singlespacing


\usepackage[cmex10]{amsmath}

\usepackage{amsthm}

\usepackage{mathrsfs}
\usepackage{txfonts}
\usepackage{stfloats}
\usepackage{bm}
\usepackage{cite}
\usepackage{cases}
\usepackage{subfig}

\usepackage{longtable}
\usepackage{multirow}

\usepackage{enumitem}
\usepackage{mathtools}
\usepackage{steinmetz}
\usepackage{tikz}
\usepackage{circuitikz}
\usepackage{verbatim}
\usepackage{tfrupee}
\usepackage[breaklinks=true]{hyperref}
\usepackage{graphicx}
\usepackage{tkz-euclide}

\usetikzlibrary{calc,math}
\usepackage{listings}
    \usepackage{color}                                            %%
    \usepackage{array}                                            %%
    \usepackage{longtable}                                        %%
    \usepackage{calc}                                             %%
    \usepackage{multirow}                                         %%
    \usepackage{hhline}                                           %%
    \usepackage{ifthen}                                           %%
    \usepackage{lscape}     
\usepackage{multicol}
\usepackage{chngcntr}

\DeclareMathOperator*{\Res}{Res}

\renewcommand\thesection{\arabic{section}}
\renewcommand\thesubsection{\thesection.\arabic{subsection}}
\renewcommand\thesubsubsection{\thesubsection.\arabic{subsubsection}}

\renewcommand\thesectiondis{\arabic{section}}
\renewcommand\thesubsectiondis{\thesectiondis.\arabic{subsection}}
\renewcommand\thesubsubsectiondis{\thesubsectiondis.\arabic{subsubsection}}


\hyphenation{op-tical net-works semi-conduc-tor}
\def\inputGnumericTable{}                                 %%

\lstset{
%language=C,
frame=single, 
breaklines=true,
columns=fullflexible
}
\begin{document}


\newtheorem{theorem}{Theorem}[section]
\newtheorem{problem}{Problem}
\newtheorem{proposition}{Proposition}[section]
\newtheorem{lemma}{Lemma}[section]
\newtheorem{corollary}[theorem]{Corollary}
\newtheorem{example}{Example}[section]
\newtheorem{definition}[problem]{Definition}

\newcommand{\BEQA}{\begin{eqnarray}}
\newcommand{\EEQA}{\end{eqnarray}}
\newcommand{\define}{\stackrel{\triangle}{=}}
\bibliographystyle{IEEEtran}
\providecommand{\mbf}{\mathbf}
\providecommand{\pr}[1]{\ensuremath{\Pr\left(#1\right)}}
\providecommand{\qfunc}[1]{\ensuremath{Q\left(#1\right)}}
\providecommand{\sbrak}[1]{\ensuremath{{}\left[#1\right]}}
\providecommand{\lsbrak}[1]{\ensuremath{{}\left[#1\right.}}
\providecommand{\rsbrak}[1]{\ensuremath{{}\left.#1\right]}}
\providecommand{\brak}[1]{\ensuremath{\left(#1\right)}}
\providecommand{\lbrak}[1]{\ensuremath{\left(#1\right.}}
\providecommand{\rbrak}[1]{\ensuremath{\left.#1\right)}}
\providecommand{\cbrak}[1]{\ensuremath{\left\{#1\right\}}}
\providecommand{\lcbrak}[1]{\ensuremath{\left\{#1\right.}}
\providecommand{\rcbrak}[1]{\ensuremath{\left.#1\right\}}}
\theoremstyle{remark}
\newtheorem{rem}{Remark}
\newcommand{\sgn}{\mathop{\mathrm{sgn}}}
\providecommand{\abs}[1]{\left\vert#1\right\vert}
\providecommand{\res}[1]{\Res\displaylimits_{#1}} 
\providecommand{\norm}[1]{\left\lVert#1\right\rVert}
%\providecommand{\norm}[1]{\lVert#1\rVert}
\providecommand{\mtx}[1]{\mathbf{#1}}
\providecommand{\mean}[1]{E\left[ #1 \right]}
\providecommand{\fourier}{\overset{\mathcal{F}}{ \rightleftharpoons}}
%\providecommand{\hilbert}{\overset{\mathcal{H}}{ \rightleftharpoons}}
\providecommand{\system}{\overset{\mathcal{H}}{ \longleftrightarrow}}
	%\newcommand{\solution}[2]{\textbf{Solution:}{#1}}
\newcommand{\solution}{\noindent \textbf{Solution: }}
\newcommand{\cosec}{\,\text{cosec}\,}
\providecommand{\dec}[2]{\ensuremath{\overset{#1}{\underset{#2}{\gtrless}}}}
\newcommand{\myvec}[1]{\ensuremath{\begin{pmatrix}#1\end{pmatrix}}}
\newcommand{\mydet}[1]{\ensuremath{\begin{vmatrix}#1\end{vmatrix}}}
\numberwithin{equation}{subsection}
\makeatletter
\@addtoreset{figure}{problem}
\makeatother
\let\StandardTheFigure\thefigure
\let\vec\mathbf
\renewcommand{\thefigure}{\theproblem}
\def\putbox#1#2#3{\makebox[0in][l]{\makebox[#1][l]{}\raisebox{\baselineskip}[0in][0in]{\raisebox{#2}[0in][0in]{#3}}}}
     \def\rightbox#1{\makebox[0in][r]{#1}}
     \def\centbox#1{\makebox[0in]{#1}}
     \def\topbox#1{\raisebox{-\baselineskip}[0in][0in]{#1}}
     \def\midbox#1{\raisebox{-0.5\baselineskip}[0in][0in]{#1}}
\vspace{3cm}
\title{Assignment-4}
\author{Ankur Aditya - EE20RESCH11010}
\maketitle
\newpage
\bigskip
\renewcommand{\thefigure}{\theenumi}
\renewcommand{\thetable}{\theenumi}

\begin{abstract}
This document contains the procedure to find value of $\sin60\degree$.
\end{abstract}
Download the python code from 
\begin{lstlisting}
https://github.com/ankuraditya13/EE5609-Assignment4
\end{lstlisting}
%
and latex-file codes from 
%
\begin{lstlisting}
https://github.com/ankuraditya13/EE5609-Assignment4
\end{lstlisting}

\section{Problem}
Show that $\sin60\degree = \frac{\sqrt{3}}{2}.$
\section{Solution}
Consider an equilateral triangle \textbf{ABC}. Since, $\triangle$\textbf{ABC} is an equilateral, all of its angles are $60\degree$. Now, The direction vector of all the sides are given as,
\begin{align}
\vec{AB}=\norm{\vec{A-B}}
\end{align}
\begin{align}
\vec{BC}=\norm{\vec{B-C}}
\end{align}  
\begin{align}
\vec{AC}=\norm{\vec{A-C}}
\end{align} 
Now for an equilateral triangle,
\begin{align}
\norm{\vec{A-B}} = \norm{\vec{B-C}} = \norm{\vec{A-C}}
\label{eq2}
\end{align}
Let, $\vec{B}$ be the origin.Hence, $\vec{B}$ = 0. Hence substituting in the equation \eqref{eq2} we get,
\begin{align}
\norm{\vec{A}} = \norm{\vec{C}} =\norm{\vec{A}-\vec{C}}
\label{e1}
\end{align}
Squaring $\norm{\vec{A}-\vec{C}}$ we get,
\begin{align}
\norm{\vec{A}-\vec{C}}^2 = \norm{\vec{A}}^2+\norm{\vec{C}}^2-2\vec{A}^T\vec{C}
\end{align} 
Substituting from equation \eqref{e1} in above equation, 
\begin{align}
\implies \norm{\vec{A}}^2 = 2\norm{\vec{A}}^2-2\vec{A}^T\vec{C}
\end{align}
\begin{align}
\implies \norm{\vec{A}}^2 = 2\vec{A}^T\vec{C}
\label{eq4}
\end{align}
\begin{tikzpicture}

\draw (0,0) node[anchor=north]{$B$}-- (4,6.9282) node[anchor=west]{$A$}-- (8,0) node[anchor=north]{$C$}-- cycle;

\draw (0,0) (0:0.75cm) arc (0:60:0.75cm);

\draw (30:1.15cm) node{$\theta$};


\end{tikzpicture}
Now by Lagrange's Identity, cross product of vectors $\vec{AB}$ and $\vec{BC}$ can be represented in terms of inner products of $\vec{AB}$ and $\vec{BC}$,
\begin{align}
\norm{\vec{AB}\times\vec{BC}}^2 = \norm{\vec{AB}}^2\norm{\vec{BC}}^2-(\vec{AB}\cdot\vec{BC})^2
\end{align}
\begin{align}
\implies (\vec{AB}\cdot\vec{BC})^2 = \norm{\vec{AB}}^2\norm{\vec{BC}}^2 - \norm{\vec{AB}\times\vec{BC}}^2 
\end{align}
\begin{align}
\implies (\vec{AB}\cdot\vec{BC})^2 = \norm{\vec{AB}}^2\norm{\vec{BC}}^2 - \norm{\vec{AB}}^2\norm{\vec{BC}}^2 \sin^2\theta
\end{align}
\begin{align}
\implies (\vec{AB}\cdot\vec{BC})^2 = \norm{\vec{AB}}^2\norm{\vec{BC}}^2 (1-\sin^2\theta)
\end{align}
Imposing the condition that $\vec{B}$ is at origin,
\begin{align}
\implies \vec{A}^T\vec{C} = \norm{\vec{A}}\norm{\vec{C}} \sqrt{1-\sin^2\theta}
\label{eq3}
\end{align}
Substituting the results of \eqref{eq4} in \eqref{eq3} and solving for $\sqrt{1-\sin^2\theta}$ we get,
\begin{align}
\sqrt{1-\sin^2\theta} = \frac{\vec{A}^T\cdot\vec{C}}{\norm{\vec{A}}\norm{\vec{C}}}
\end{align}
From equation \eqref{e1}, $\norm{\vec{A}} = \norm{\vec{C}}$.
\begin{align}
\implies \sqrt{1-\sin^2\theta} = \frac{\vec{A}^T\cdot\vec{C}}{\norm{\vec{A}}^2}
\label{e2}
\end{align}
Imposing the results of \eqref{eq4} in \eqref{e2} we get,
\begin{align}
\implies \sqrt{1-\sin^2\theta} = \frac{\vec{A}^T\cdot\vec{C}}{2\vec{A}^T\vec{C}}
\end{align}
\begin{align}
\implies \sqrt{1-\sin^2\theta} = \frac{1}{2}
\end{align}
Squaring both the sides and solving for $\sin\theta$ we get,
\begin{align}
\therefore \sin^2\theta = 1-\frac{1}{4}
\end{align}
\begin{align}
\implies\sin\theta = \frac{\sqrt{3}}{2}.
\end{align}
$\therefore \theta = 60\degree$ 
\begin{align}
\implies \sin60\degree = \frac{\sqrt{3}}{2}.
\end{align}
\end{document}